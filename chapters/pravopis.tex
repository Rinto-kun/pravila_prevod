\chapter{Правопис}

\section{Правопис при писане на субтитри}

\subsection{Тире на първи ред при наличие на диалог}
В английските субтитри може да слагат тире срещу реплики на всеки от участниците в разговора, но в българските поставяме такова само пред репликата на втория участник (тоест, пред втория ред на субтитъра).

\subsection{Дължина на субтитрите}
Винаги проверявайте CPS до 15, и символите до 37, на субтитрите, чрез AegiSubs и задължително редактирайте и превеждайте с нея. Ако не сте наясно с това, питайте ръководителя си.

\subsection{Бележки на преводача}
Всякакви бележки от преводача, разяснения и лични впечатления под и над субтитър са забранени.

\subsection{Звукоподражания}
Забравете за всякакви звукоподражания от рода на Ъъъъ, ахх, мхм, тц, амиии, както и всевъзможните варианти с удължаване на гласни: майкооо, божеее, даааа, неее и т.н. Те са напълно излишни, защото ние и без друго си ги чуваме.

\subsection{Числата}
Прието е числата над 10 хиляди в българските субтитри да се изписват с интервал пред всеки три нули. Например: 23 000; 115 030; 1 500 000 и т.н. В английските субтитри обикновено на мястото на интервалите поставят запетаи, но вие не го правете. Примери за грешно изписване на числата са: 23000; 115.030; 1,500,000 и т.н. Противно на това правило, часовете се отбелязват на принципа "час - запетая - минути" ето така: 23,00 ч.; 08,15 ч.; 12,40 ч.; 16,30 ч. и т.н.

\subsection{Неправилното членуване}
Може би най-често срещаната грешка не само в субтитрите, но и във форума. Какъв по-добър пример от ,,Филма беше много хубав''? Правилният вариант в случая е ,,Филмът беше много хубав''. За мен най-лесният начин за откриването на пълния член е намирането на подлога. Подлогът (който върши действието) или думите, които го поясняват (прилагателните), винаги получават пълен член.

Пример:
Дърварят (подлог) отсече дървото.
Старият (пояснение на подлога) дървар (подлог) отсече дървото.

Има още едно правило, което може да ви е от помощ: думите, които се намират непосредствено след предлог (в, на, под, над, зад, пред, след и т.н.) винаги се пишат с кратък член.

Пример:
След (предлог) субтитъра няма кавички.

Има и изключения: не се членуват прякорите. Например ,,Хари Потър и Нечистокривния принц''.

\subsection{Избягването на ,,ok?''}
в края на изречението също е препоръчително, да не кажа задължително. Американците явно са влюбени в него и го използват постоянно, но това не прави почти всяко изречение в субтитрите въпросително. Когато е такова, в българския език съществува прекрасното ,,нали?'' или ,,ясно?''. Разбира се, според контекста, преводът на "ok" може да бъде и съвсем различен.

\subsection{Тя, той, ние, вие, това, онова, просто и т.н.}
присъстват в почти всяко изречение от английската реч, но прекомерната им употреба в българските субтитри утежнява изречението или накратко дразни. Ако без тях не се променя смисълът на изречението, избягвайте ги.

Пример:
Ще занесем папката по-късно.
- Аз мисля, че просто трябва да занесем това веднага.

Ще занесем папката по-късно.
- Мисля, че трябва да го направим веднага.

\subsection{Слято писане}
Погрешно е разделянето на оттук, оттам, откакто, откога, отколкото, дотам, досега, засега, доколкото, докъдето, натам, насам, наново, отпред, отначало и т.н., но зависи:

Пример:

Затова може да се изписва и като ,,за това'', ако се пояснява нещо - лице, пред и т.н...

,,За това колело няма части''

\subsection{Препинателни знаци}
Препинателните знаци в края на изреченията правят голямо впечатление, ако са поставени грешно. Не оставяйте знаците от оригиналните субтитри, те невинаги са верни. Освен това избягвайте комбинации от препинателни знаци или натрупването на два или три препинателни знака от един и същи вид (пр. ???, !!! и т.н.).

\subsection{Тавтология.}
Невинаги може да бъде избегната, а и понякога самият филм налага употребата й, но се старайте да не повтаряте едни и същи думи в непосредствена близост или в съседни субтитри. Тук на помощ ви идва добрият стар синонимен речник.

\subsection{Отделни проблеми}
\begin{enumerate}
    \item Да няма изречения завършващи с многоточие и въпросителен или възклицателен знак; Пример - "Какво...?,, ''Няма ли...?,, ''Не...!" Пише се само многоточие или препинателен знак!
    \item Избягвайте изречения започващи от сорта ,,Та, какво...'', ,,Но, каквото...''
    \item ,,ТОКУ-ЩО'' се пише с тире! Никога не го забравяйте! Един-единствен, веднъж-дваж, два-три, току-виж, раз-два...
    \item Засягам този тип грешки в отделна точка, тъй като са нечовешки често срещани - ,,не знам'', ,,не мога'', ,,не искам'' и от този сорт, не се пишат слято!
    \item ,,тебе'' и ,,мене'' са вече минало, използват се ,,теб'' и ,,мен''. Възможно е използването им единствено в песни, цитати на старинни текстове и т.н. Също така вече не се използват и глаголи от сорта на ,,отидемЕ'', ,,влеземемЕ'', ,,говоримЕ'', ,,качимЕ'' и т.н. Подобни форми са отживелица. Никакви ,,Е''-та накрая!
    \item След ,,тире'', ,,точка'', ,,запетая'' и други знаци се поставя интервал. Това не важи за случаите ,,г-жо'', ,,г-н'', ,,д-р'', ,,по-важен'', ,,най-голям'', ,,т.н.'' и други.
    \item На съкращения на мерни единици - км, см, м, мм, л, мл, кг, т, км/ч и т.н. не се поставя точка.
    \item Имена на плавателни съдове, били те по вода, въздух или в космоса, се пишат в кавички - ,,Титаник'', ,,Радецки'', ,,Аполо'', ,,Ксантиум'', ,,Еър Форс''...

    \item,,Със'' се пише само пред думи започващи с букви ,,С'' и ,,З'', пред всички останали е само ,,С''. ,,Във'' се пише само пред думи започващи с букви ,,В'' и ,,Ф'', пред всички останали е само ,,В''.
    
    \item Не съществуват думи като В предвид, на предвид и т.н., пише се само предвид... 
\end{enumerate}

\subsection{Пълен и кратък член}
Когато е възможно да заместите съмнителната дума за вас в изречението с ,,той'' - тогава е пълен член - ,,ят'', ако не става е кратък член. Преглеждайте го неколкократно, за да не се излагате. Адски дразнещи са подобни грешки, направо очеизбождащи. Мнозинството ,,преводачи'' смятат, че разбират това правило, но това не е така. Пълен член се пише и винаги след спомагателния глагол ,,съм''. (Аз съм учителят му. Той е добрякът... и т.н.) И пълен член не се пише след предлози (с/със, без, на, в/във, до и т.н.).

\subsection{Обръщения и запетайки}

Когато актьорът се обръща към друг с името му, се поставя запетайка (група от хора /Здравейте, приятели/)

\begin{verbatim}
    Шели, колко пъти да ти повтарям
    да не оставяш навън джаджите си?

    Всичко наред ли е, Джон?

    Добре ли се справих, момчета?

    Как си, малкото ми ангелче?

    Какво става, по дяволите?

    Ах ти, негодник!
\end{verbatim}

При ситуации ,,Да, и ще се върнем.'', ,,Какво, проблем ли е?'', ,,Защо, кой пита?'', ,,Кой, аз ли?'' се поставя запетая.

\subsection{Поставяне на кавички.}
Когато имаме субтитър от сорта:

Търсете ,,Филип Грийнбърг''. - в този случай кавичките са преди точката.

Ако се цитира цяло изречение - тогава цялото е в кавички:

\begin{verbatim}
    204
    00:18:17,246 --> 00:18:23,344
    "Неговият необятен гений е надеждата
    ни да донесем мир в галактиката си."
\end{verbatim}


- В случай, че мисълта е по-дълга от един субтитър се слагат кавички единствено в началото и края:


\begin{verbatim}
    208
    00:18:37,809 --> 00:18:42,913
    "Шели? Дойдоха да те отведат
    в бъдещето.

    209
    00:18:46,318 --> 00:18:51,554
    Не забравяй да си вземеш бельо!"
    - Добре, така е добре.
\end{verbatim}


\subsection{Вас, вие, ви}
Вие, вас, ви, в субтитрите е прието да се пише с малка буква, докато в официален текст и документи, според лицето, с главна. 

\subsection{Пренасяне на нов ред}
Как се пренася чрез \textbackslash N, накратко на кои места... (На субтитър имате право само един път да пренасяте чрез \textbackslash N , а не от сорта ,,Той отиде там,\textbackslash N за да ти каже,\textbackslash N че си...'' - ТОВА Е ГРЕШНО! Само 1 път се ползва \textbackslash N на субтитър!)

\begin{enumerate}
    \item СЛЕД запетая, точка, удивителен, въпросителен, многоточие...
    \item ПРЕДИ следните съюзи и глаголи, ТОЕСТ всичко изброено отива на втори ред:

    - с, на, да, до, за, при, към, над, до, под, в/във, с/със, върху, между...
    - мога да, искам да, смятам да...

    \item не се цепят частици ,,не'' от глаголи, когато се пренася на втори ред;
    - не се цепят частици ,,се'' от глаголи, когато се пренася на втори ред. 
    Пример: 
    
    "Училището ми. При последната мъгла
    се скъса мрежата и те нахлуха."
    
    \item да не се делят имена - Шелдън Купър - да бъдат на един ред;
    \item да не се дели прилагателно от съществително;
    \item но когато е невъзможно се цепи както дойде;

\end{enumerate}

\section{Пунктуация}
\subsection{Ограждаща запетая}
Ограждащата запетая отделя синтактични единици със свое логическо или фазово ударение: обръщения, вметнати думи и изрази, обособени части, подчинени определителни изречения.

\begin{enumerate}
    \item Обръщението, ако е в средата на изречението, винаги се загражда от двете страни със запетая (Не е хубаво, синко, да забравяш важни неща); ако е в началото, запетаята е след него (О, майко моя, родино мила, защо тъй жално, тъй милно плачеш), а ако е в края пред него (Моя ли е грешката, звезди?)
    \item Вметнати думи и изрази се отделят със запетая. Винаги вметнати и затова отделени със запетая са глаголи и някои противопоставителни или служещи за подреждане изрази: разбира се, изглежда, тъй да се каже, мисля, да кажем, напротив, обратно, от една страна, от друга страна, първо, второ, с една дума:
    \textit{Според Хаджийски апостолите са си поставили две главни цели: първо, да накарат хората да забравят; второ, да се създаде безусловна вяра в успеха. От една страна, той има право, но от друга страна, дали всичко е така, както казва? Убеден е, че тя, неговата стопанка, нищо не е скрила. С една дума, той живееше безгрижно като птица.
    Неговият стил е, така да се каже, народноразказвателен. Ако бащата, бог да го прости, не му е оставил нищо, ти ще оставиш. Ти, разбира се, ще дойдеш. Ти, напротив, няма да дойдеш.}
    \item Не се отделят със запетаи въвеждащите изрази и думи: според мен, за жалост, очевидно, по мое мнение, обаче, например, може би, наистина, вероятно, очевидно, сякаш, като че ли, всъщност, по такъв начин, по всяка вероятност, следователно, значи. \textit{
    Това разумно предложение обаче предизвиква възражения. Този факт например доказва, че нещо в организацията не е в ред. Интересно наистина защо не се съгласиха. Това всъщност е най-доброто решение. Тя като че ли не е съгласна.
    Ако просто изречение вътре в рамките на сложното започва с обаче, следователно, сякаш, значи и др., пред тези думи се пише запетая: Това обстоятелство дава известни надежди, обаче не бива да се разчита напълно. Разделиха се с прегръдки и целувки, сякаш нямаше да се видят утре.}

    \item Вметнати части и междуметия се отделят със запетая: \textit{Дa, и него поканих. О, колко има да чакаш! Хайде, какво чакаш!}
    \item Обособените части винаги се отделят със запетаи. Правилото е без изключения и се отнася дори за случаите, когато след обособената част има съюз или съюзна дума, пред които е известно, че не се пише запетая (и, да, къде, как, защо). Обособените части интонационно се отделят с пауза, която графично се представя със запетая. Обособяването се налага с цел подчертаване на логическия акцент или по стилистични причини. Обособените части могат да се трансформират в подчинени изречения – определителни или обстоятелствени, които също се отделят със запетая.
\textit{Археологията е сравнително млада наука, възникнала в края на XVIII и началото на XIXвек. Археологията е сравнително млада наука, която е възникнала в края на XVIII и началото на XIX век.
Изпод черното покривало, скриващо лицето на жената, святкаха млади очи. Изпод черното покривало, което скриваше лицето на жената, святкаха млади очи.
Търсейки истината, той всъщност откриваше безбройните проявления на лъжата. Докато търсеше истината, той всъщност откриваше безбройните проявления на лъжата.
Когато пред обособената част, започваща с причастие или деепричастие, има едносричен неударен съюз, се допуска да не се пише запетая между съюза и обособената част: Но изградил си веднъж авторитет, човек не бива да разчита само на него. Седеше спокойно на масата и усмихвайки се, раздаваше съвети.}

    \item Подчинени определителни изречения винаги се ограждат със запетая.
\textit{Лъжецът, който разказваше, се сгуши в подплатения с кожа елек и замлъкна. Много хора работят това, което не им е по сърце, и страдат. Слушах музиката, която гърмеше, не защото ми харесваше, а защото нямах друг избор. Срещнах учителката, под влияние на която се записах да следвам химия.}
\end{enumerate}

\subsection{Разделяща запетая}

\begin{enumerate}
    \item Повторените части се отделят със запетая: Тежко, тежко, вино дайте! Покой, покой ми трябва поне за минута.

\item Еднородни части (определения, подлози, допълнения, обстоятелствени пояснения) в рамките на простото изречение се отделят със запетая: Говореха за фронта, за ранените и убитите, за глада, за пленниците. Между многото му слабости ловът беше най-любимата, най-властната. С годините това място ми се струва по-високо, по-стръмно, по-труднодостъпно.

\item Пред съюзите а, но, че, макар че, въпреки че, само че, защото, щом, преди да, след като, за да, обаче, когато с тях се въвежда изречение, се пише запетая. При сложни съюзи като въпреки че, макар че, само че запетая се пише пред целия съюз, а не пред втората му съставна част. Пред съчетанията при условие че, при положение че, когато са употребявани като съюзи, запетая се пише пред цялото словосъчетание: Експериментът протича нормално, при условие че всички елементи са едновременно налице. Слушаха го, макар че не го обичаха. Не разбираше, въпреки че най-добросъвестно се стараеше.

\item При повторени съюзи запетая се пише пред втория. Нито се обърна, нито каза нещо. Дойдоха и майка му, и баща му. Не поглеждаше ни наляво, ни надясно.

\item При съотносителни съюзи запетая се пише пред втория. Взеха както летни, така и зимни дрехи. Чухме колкото глупости, толкова и умни неща. Глупост ли е туй, или раболепие? С мен ли ще говориш или с майка ми? Провериха не само гардероба, но и всички шкафове в къщата. Не разбраха дали да чаката, или да тръгват.

\item Пред съюзите и и или запетая не се пише, ако са единично употребени. Ако са повторени, пред втория съюз се пише запетая. Има случаи, когато е необходимо вглеждане в структурата на изречението, защото зад привидни еднаквости се крият различни смислово-синтактични отношения. \textit{Дойдоха и майка й, и баща й и я отведоха. Пред втория съюз и се пише запетая, защото свърза еднородни подлози. Пред третия съюз и запетая не се пише, защото с него еднократно се въвежда второ главно изречение.}
Когато съюзът и е в съчетание с то или с да – и то, и да, пред тези съюзи запетая винаги се пише: \textit{Жена, и то хубава жена се беше изправила на пътя ми. Хора, и то много хора бяха дошли. Няма да дойдат, и да ги каниш. Не му се сърдеха, и да викаше, и да се караше.}

\item Пред съюз да запетая не се пише, когато въведеното с него изречение е главното: Кажи ми да дойда. Ела да те видя. Ако подчиненото изречение, което започва със съюз да, е в началото, след него се пише запетая: Да дойдат, им кажи. Да се върнат, не помислиха.
Когато да е част от сложен съюз, запетая се пише пред целия съюз – за да, преди да, без да: \textit{Сбогуваха се, преди да се качат във влака. Бързаха за да стигнат навреме. Отговориха, без да помислят.}
Пред да се пише запетая, когато е повторен или употребен вместо ако или за да: Обичаше да чете, да слиша музика, да рисува. \textit{Щеше да можеш, да (=ако) беше се упражнявал. Изведоха невестата, да (=за да) я видят сватбарите.}

Ако подчиненото изречение, започващо със съюз да, пояснява показателното местоимение това, пред да се пише запетая: Няма нищо по-хубаво от това, да търсиш зелената трева на пустината.

Когато подчиненото изречение, започващо със съюз да, стои след съществително, придружено от показателно местоимение, пред съюза да се пише запетая: Тази идея, да съберем всички пари на едно място, не ми харесва. Този нейн каприз, да си облече най-хубавата рокля, й струва скъпо.

\item Пред въпросителна дума (къде, как, кой, кога, защо) запетая не се пише. Попитах го къде отива. Интересувам се кой ще дойде. Не знам защо ще идва.
Ако подчиненото изречение, което започва с въпросителна дума, пояснява показателното местоимение това, пред въпросителната дума се пише запетая: Не ме интересувашe, това къде отиваш. Това, дали ще ми кажеш, не е важно в момента. Престанах да мисля за това, защо постъпва така с мен.


\item Пред относителните местоимения (който, където, когато, чиито, както и пр.), с които се въвежда подчинено определително изречение, се пише запетая. Ако пред местоимението има подлог, запетаята се пише пред подлога: Котаракът бил единственото същество, което го чакало вкъщи. Усмихваше се като човек, на когото му е все едно дали ще го приемат. Дълго се взирах в селото, над което се виеше черен облак дим.
Пред относително местоимение запетая не се пише в следните случаи:

\begin{enumerate}
    \item Когато местоимението е в състава на устойчиво словосъчетание, което може да се заместо с една дума: Ще има колкото трябва = достатъчно. Ям каквото ми падне = всичко. Отивам където ми видят очите = някъде.
    
    \item Когато с относително местоимение се въвежда подчинено допълнително изречение. В простото изречение Ще попитам някого позицията на допълнението е заета от някого. Ако тази позиция се заеме от подчинено изречение, то е подчинено допълнително: Ще попитам когото срещна. В простото изречение Ходих при него допълнението е при него. Ако се заместо с подчинено изречение, то е подчинено допълнително: Ходих при когото ми каза. Аналогични са трансформациите Излизам с приятен човек = Излизам с когото ми е приятно.
    \item Когато подчиненото подложно изречение, започващо с относително местоимение, е след главното, пред подчиненото изречение запетая не се пише; ако е пред главното – след него се пише запетая. В простото изречение Не трябва да не идват болните подлогът е болните. Позицията на подлога може да се заеме от подчинено подложно изречение: Не трябва да не идват които са болни; но Които са болни, не трябва да идват. В изречението Не трябва да говори незнаещият подлогът е незнаещият. Неговата позиция може да се заеме от подчинено подложно изречение: Не трябва да говори който не знае; но Който не знае, не трябва да говори.
    \item Когато пред относителното местоимение се намират думите само, едва, чак, даже, тъкмо, именно, точно, сигурно, може би запетая не се пише: \textit{Ще му проговоря едва (чак, само) когато ми се извини. Върнах се точно (именно)защото те обичам. Ще им дам точно колкото искат.}
\end{enumerate}

\item Пред отрицателната частица не в съчетание с относително местоимение запетая не се пише:\textit{ Дадох не колкото искаше, а колкото имах. Попитах не защото не знам, а за да го проверя.}
\item 
Ако пред подчиненото изречение, започващо с не+относително местоимение, свършва обособена част или подчинено определително изречение, запетая се пише: \textit{Попитах човека, седнал в ъгъла, не защотото очаквах правилен отговор, а за да привлека вниманието му. Отворих прозореца, който гледаше към улицата, не когато ми каза, а когато реших.}

\item При съчетаване на два съюза, пред всеки от които при единичната му употреба се пише запетая, къде и дали ще се пише запетая, зависи от спецификата на първия съюз.
\begin{enumerate}
    \item Ако първият съюз е едносричен и без собствено ударение, запетая се пише само пред него, т.е. пред втория не се пише: \textit{Сигурна бях, че ако кажа, ще ме обвинят}. В тези случаи е важно да се определи къде свършва подчиненото изречение, въведено с втория съюз, за да се отдели то със запетая. Цитираното сложно изречение се състои от три изречения: сигурна бях (главно); че ще ме обвинят (подчинено допълнително); ако кажа (подчинено обстоятелствено за условие към подчиненото изречение). При структурирането на сложното изречение подчиненото обстоятелствено изречение за условие ако кажа трябва да се отдели със запетая в края. В изречението \textit{Тръгнах, но за да не закъснея, взех такси} първото главно изречение е тръгнах, второто главно е но взех такси, а изречението за да не закъснея е подчинено обстоятелствено за цел и след него трябва да се пише запетая.
    \item Ако първият съюз е многосричен и със собствено ударение, запетая се пише и пред двара съюза. \textit{Това са хората, които, вместо да се радват, плачат. Селяните се страхуваха, защото, когато заседаваше съдът, винаги се случваха неприятни работи.}

\end{enumerate}


\item Когато със съюз като се въвежда подчинено изречение, пред него се пише запетая. Ако със същия съюз се въвежда сравнение, пред него не се пише запетая: Господин управителят, като подписа книжката, отряза голямо парче тиква и му го предложи.
\end{enumerate}