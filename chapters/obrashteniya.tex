\chapter{Японски обръщения}
В тази част смятам да поясня малко повече за обръщенията, които се използват
в японския език и \textbf{как} и \textbf{дали} да ги \textbf{превеждаме}. Като начало трябва да се каже,
че ако се превеждат тези обръщения, никак не е уместно да се ползват множеството обръщения, навлезли от английски, френски и т.н. от сорта на "\textbf{сър}", "\textbf{мистър}", "\textbf{мадам}" и т.н.,
\textbf{освен ако контекстът изрично не го изисква}. Също трябва да спомена, че някои от обръщенията
са \textbf{непреводими} и ако държите да превеждате обръщенията, просто ще се наложи да ги пропускате.

\section{Често срещани}
\begin{longtable}{|p{0.2\textwidth}|p{0.8\textwidth}|}
\hline
\textbf{Наставка} & \textbf{Превод} \\ \hline
\endfirsthead
\hline
\endhead
\hline
\endfoot

\textbf{-san}& г-н, г-жа, г-жица \\
\textbf{-chan/-kun}& практически \textbf{непреводимо}, оставяте само името \\
\textbf{-sama/-dono}& господарю, господарке (понякога се използват с ироничен привкус и тогава може да използвате различни вариации като \textbf{,,всемогъщи''}, \textbf{,,вездесъщи''} и т.н., но \textbf{само в шеговит контекст}) \\
-\textbf{sempai}& може да се каже, че е непреводимо, евентуално може като \textbf{,,старши''}, но \textbf{само ако става въпрос за отношения на работното място}, например \\
\textbf{-kouhai}& по-неопитен и по-низш в йерархията, няма адекватен превод \\
\textbf{-sensei}& евентуално може да се преведе като \textbf{,,учителю''}, но съветвам и тук по-скоро да се оставя само името, тъй като \textbf{това обръщение се използва и за лекари, професори и т.н.} \\
\textbf{-senshuu}& за спортисти, евентуално като \textbf{,,състезател''}, или в зависимост от спорта, за който става въпрос. \\

\hline\caption{Основни обръщения в японския език}

\end{longtable}

\section{Фамилни имена}

\begin{table}[htbp]
    \centering
        \begin{tabular}{|p{0.15\textwidth}|p{0.85\textwidth}|}
            \hline
            \textbf{Наставка} & \textbf{Превод} \\ \hline
            \textbf{okaa-san / kaa-san}&майка, мамо\\ 
        \textbf{otou-san / tou-san}&татко, тате\\ 
        \textbf{onee-san / nee-san}&кака, како (както и в българския, може да се обръщат с "како" и към по-голямо момиче, което не е в семейството, но им е близко)\\ 
        \textbf{onii-san / nii-san}&батко, бате (казаното за \textbf{onee-san} важи и тук)\\ 
        \textbf{otouto}&по-малък брат, братче (тъй като в българския нямаме определена дума за това ви предлагам да не го превждате навсякъде където се споменава, а да го замествате с име или местоимение)\\ 
        \textbf{imouto}&по-малка сестра, сестричка (казаното за "otouto" важи и тук)\\ 
        \textbf{ojii-san / jii-san}&дядо (спокойно се използва и за хора извън семейството, както и в български)\\ 
        \textbf{obaa-san / baa-san}&баба, бабо (и това също се използва за хора извън семейството)\\ 
        \textbf{oji-san}&чичо\\ 
        \textbf{oba-san}&леля, лельо (и ,,чичо'', и ,,лельо'' пак са обръщения, които могат да се използват към хора, които не са роднини)\\ 
        
        \hline
        \end{tabular}
    \caption{Фамилни имена}
    \label{tab:familynames}
\end{table}

Както е видно, с изключение на \textbf{,,майка''} и \textbf{,,татко''}, всички други могат да бъде използвани и извън семейството.
Съветвам ви в тези случаи \textbf{не винаги да ги превеждате}, а да използвате \textbf{имена} и \textbf{местоимения} като \textbf{заместители}, където е възможно, защото японците не се уморяват да ги повтарят и \textbf{на български понякога звучи тромаво или странно.}


\section{Армейски звания}
Някои от по-често употребяваните звания в армията са описани в таблиците по-долу.
\begin{table}[htbp]
    \centering
    \begin{tabular}{|m{9em}|m{9em}|m{16em}|}
        \hline
        Японски & Английски & Български\\
        \hline
        \textbf{Nishi \begin{CJK*}{UTF8}{song}
            (二士)
        \end{CJK*}} & Private 2nd Class & \textbf{редник}\\ 
        \textbf{Isshi \begin{CJK*}{UTF8}{song}
            (1士)
        \end{CJK*}}& Private 1st Class & \textbf{редник}\\ 
        \textbf{Shichou \begin{CJK*}{UTF8}{song}
            (士長)
        \end{CJK*}} & Leading private & \textbf{ефрейтор}\\ 
        \textbf{Sansou \begin{CJK*}{UTF8}{song}
            (三曹)
        \end{CJK*}} & Sergeant & \textbf{младши сержант} (най-младши командир)\\ 
\textbf{Nisou \begin{CJK*}{UTF8}{song}
    (二曹)
\end{CJK*}} & Sergeant first-class & \textbf{сержант} (младши командир)\\ 
\textbf{Issou \begin{CJK*}{UTF8}{song}
    (一曹)
\end{CJK*}} & Master sergeant & \textbf{старши сержант} (старши командир)\\ 
\textbf{Souchou \begin{CJK*}{UTF8}{song}
    (曹長)
\end{CJK*}} & Master sergeant; sergeant major & \textbf{старшина} (най-старши командир) \\
\hline
    \end{tabular}
    \caption{Нисши звания в японските отбранителни сили – JSDF
    (сухопътни войски)}
\end{table}

\begin{table}[htbp]
    \centering
    \begin{tabular}{|m{9em}|m{9em}|m{16em}|}
        \hline
        Японски & Английски & Български\\
        \hline
        \textbf{Jōtōhei Kimmusha \begin{CJK*}{UTF8}{song}
            (上等兵勤務者)
        \end{CJK*}} & Acting Senior Private & \textbf{Редници}\\ 
        \textbf{Nitōhei \begin{CJK*}{UTF8}{song}
            (二等兵)
        \end{CJK*}}& Private 2nd Class  & \textbf{редник} (в съвременната бълг. армия няма редници от 2-ри и 3-ти клас)\\ 
        \textbf{Ittōhei \begin{CJK*}{UTF8}{song}
            (一等兵)
        \end{CJK*}} & Private 1st Class  & \textbf{редник}\\ 
        \textbf{Gochō Kimmu jōtōhei \begin{CJK*}{UTF8}{song}
            (伍長勤務上等兵)
        \end{CJK*}} & Junour Corporal & \textbf{Ефрейтори} \\
        \textbf{Jōtōhei \begin{CJK*}{UTF8}{song}
            (上等兵)
        \end{CJK*}} & Seniour Private & \textbf{ефрейтор} \\
        \textbf{Heichō  \begin{CJK*}{UTF8}{song} 
            (兵長)
        \end{CJK*}} & Lance Corporal & \textbf{ефрейтор}  \\       \textbf{Gochō \begin{CJK*}{UTF8}{song}
            (伍長)
        \end{CJK*}} & Corporal & \textbf{ефрейтор} \\
        \hline
    \end{tabular}
    \caption{Нисши звания в японската армия през ВСВ (сухопътни войски)}
\end{table}

\begin{table}[htbp]
    \centering
    \begin{tabular}{|m{9em}|m{9em}|m{16em}|}
        \hline
        Японски & Английски & Български\\
        \hline
        \textbf{Gunsō \begin{CJK*}{UTF8}{song}
            (軍曹)
        \end{CJK*}} & Sergeant & \textbf{Сержант}\\ 
        \textbf{Sōchō \begin{CJK*}{UTF8}{song}
            (曹長)
        \end{CJK*}}& Sergeant Major  & \textbf{Старшина} \\
        \hline
    \end{tabular}
    \caption[Сержантски звания]{Сержантски звания. В съвременната българска армия имаме младши сержант (най-младши командир); сержант (младши командир); старши сержант (старши командир); старшина (най-старши командир).}
\end{table}

\section*{Бележки}
\begin{itemize}
    \item \textbf{heichou} e нисше звание - редник или ефрейтор (ако е във флота - матрос или старши матрос). Най-общо войници.
    \item \textbf{taichou} е сержант или старшина, командващ взвод или рота.
    \item \textbf{buntaichou} е командир на екип или отряд в пожарната, обаче при нас пожарникарите май нямат звания и мисля, че еквивалент няма.
    \item \textbf{danchou} трябва да е лидер (на нещо си), упълномощено лице, водач - на група, на партия, на фирма и т.н.
    Доста широко понятие и не намерих да е свързано с нещо конкретно.
\end{itemize}


