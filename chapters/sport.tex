\chapter{Спортни термини}
\section{Бейзбол}
Бейзболът се играе от 2 отбора, съставени от по 9 играча. Всеки отбор е воден от мениджър. Играе е се на специално поле. Съдийства се от 1 до 4 съдии. Основната цел на всеки отбор е да отбележи повече точки. В бейзбола не може да има равен мач, играе се до победа на единия отбор.

Полето се състои от две основни части: феър и фал територия. Линиите, които разделят двете територии се наричат фал линии. Те са с дължина взависимост от разстоянието до оградата на игрището - от 98 до 122 м.

Феър територията се дели на "инфилд" (оцветеното в жълто) и "аутфилд" (оцветеното в тъмнозелено). Инфилдът е площта на квадрата (страната му е 27,42 м), на чийто ъгли са поставени базите, и площта, заградена от дъгата на инфилда. Аутфилдът е останалата част от феър територията. Фал територията (оцветеното в светло зелено) е остатъкът от игрището между фал линиите и предпазните странични огради.

На всеки ъгъл от квадрата на инфилда е закрепена по една база. Началният ъгъл, който е и ъгъл на феър територията се нарича "хоум" (от англ. "къща") поради формата му, която наподобява силует на къща. Това е една от най-важните бази. От нея започва играта, на тази база се застава, за да се удря (батира) топката. След обиколка на целия квадрат стъпването върху хоума води до отбелязване на точка за отбора, който батира.

Останалите бази, броени обратно на часовниковата стрелка се наричат съответно 1-ва, 2-ра и 3-та база. Те са изработени от дунапрен или друга подобна мека материя и приличат на квадратни възглавнички (страната им е 38 см).

В играта се ползват бухалка, ръкавица и топка. В България се ползват само алуминиеви бухалки (одобрени от федерацията). Те имат различни дължини и дебелини, максималните размери са: дължина - 107 см, дебелина - 7 см.

Топката е изработена от корков център, обвит от гума, върху която са намотани вълнени конци. Върху тях са зашити две кожени парчета. Тя е тежичка (от 143 до 147 гр. и обиколка 23-23,7 см.), поради което играчите са задължително с каски.

Ръкавиците се изработват от кожа, големината им е различна според поста, на който се ползват. Специална е ръкавицата за кетчера и на първия бейзмен. Останалите ползват стандартни ръкавици.

Oтборът се състои от играчи със следните позиции:

\begin{quote}
    1. Девет играчи: питчер (f1); кетчър (f2); първи бейзмен (f3); втори бейзмен (f4); трети бейзмен(f5); шортстоп (f6); ляв аутфилдер (f7); централен аутфилдер (f8); десен аутфилдер (f9).
\end{quote}

    \begin{longtable}{|m{6em}|m{25em}|}
        \hline
        Термин & Превод и описание\\
        \hline
Appeal play &Апел – Обръщение от страна на треньор или играч към съдия по време на игра с цел да се обърне внимание на извършено нарушение на правилата.\\ 
Backstop &Бекстоп – Ограждението намиращо се на 18 м. зад хоума.\\ 
Ball &Бол - Лошо хвърляне. Недобро хвърляне на питчера, регистрирано от съдията на хоума, при което топката преминава извън страйк зоната на играча.\\ 
Balk &Балк - неправилно или непозволено хвърляне на питчера; в случай, че има играчи по базите (рънерите), те напредват по една база напред.\\ 
Base &База\\ 
Base on balls or walk &База за боли - "награда" за батера, която се изразява в напредване до 1-ва база без риск за изгаряне. Това е следствие на 4 лошо хвърлени топки от питчера.\\ 
Baseline &Базова линия – Линиите, които описват квадрата между базите.\\ 
Batter &Батер, батиращият\\ 
Bottom of the inning &края на ининга\\ 
Bunt &Бънт – Топка, ударена с подлагане на бухалката (т.е. без да се замахва). Целта е топката да се удари леко и да остане в инфилда.\\ 
Catch/Catcher &Улавяне/Кетчър – Хващане на топката от въздуха, преди тя да е докоснала земята или огражденията на игрището. Батера, който е ударил тази топка се брои за аут (изгорен).\\ 

Coach &Коуч - Помощник на треньора, който застава на определено място край игрището за да направлява рънерите със знаци или думи.\\ 
Curve ball &Кърв бол или казано на български "ниска топка"\\ 
Dead ball &Мъртва топка - положение, при което играта е временно прекъсната, никой няма право да играе с топката и тя трябва да се намира на възвишението на питчера.\\ 
Double &Дабъл - добър удар, при който батерът успява безпроблемно да напредне до 2-ра база.\\ 
Double play &Дабъл плей – Ситуация, при която отбора в защита успява от един удар на противника да овладее топката и да изгори двама играча последователно.\\ 
Dugout &Дъгаут -Пейката за резервните играчи.\\ 
Fair Ball &Топка феър/Феърбол – Топка, ударена във феър територия. Зачита се за удар и всеки играч има право и може да се предвижва към следваща база, за да достигне отново хоума и да отбележи точка. Играчите напредват докато топката не се върне в инфилда и възникне опасност да бъдат изгорени, като ги хванат извън база.\\ 
Foul Ball &Топка фал/Фалбол – Топка, ударена от батера и преминала във фал територията. Зачита се за страйк, но не се зачита за удар. Никой играч в нападение не може да ползва тази топка, за да напредне към следваща база.\\ 
Hit &Хит - удар, направен по такъв начин, че батерът да си осигури достатъчно време да стигне до някоя база без възможност да бъде изгорен. Обикновено са силни далечни удари или силни удари по земя между двама противникови играчи.\\ 
Home Run &Хоумрън – Силен удар, който прелита над цялото игрище (феър територия) и пада извън него. Първия досег с земята е извън игрището зад голямата дъга на феър територията. При това положение на батера се дава възможност безпрепятствено да обиколи всички бази и да отбележи точка. Ако има други негови играчи – рънери, те биват избутвани от него по този начин и те също бележат точки, след като достигнат до хоума.\\ 
Home &Хоум, началната и крайна база\\ 
Home run &Хоумрън - Силен и далечен удар\\ 
Infield Fly &Инфилд флай – Ситуация, при която има играчи на 1-ва и 2-ра база или 1-ва, 2-ра и 3-та база и батера удари висока топка, падаща в рамките на инфилда. В този случай батерът се обявява веднага в аут още преди топката да е паднала на земята, а рънерите не са принудени да напредват по базите.\\ 
Inning &Ининг – Част от играта, по време на която всеки един от двата отбора се изрежда един път в защита и един път в нападение.\\ 
Interference &Намеса – Положение, при което някой играч се опита да попречи извън позволеното от правилата на противников играч да извърши някакво действие (удряне на топката, достигане до базата и др.)\\ 
Line Drive &Линия – Остра права топка, летяща бързо и успоредно над земята.\\ 
Mitt &Ръкавица (среща се и като baseball glove)\\ 
Out &Аут (изгаряне) - действия на отбора в защита, при които играч от противниковият отбор се отстранява от играта. Например: ударена топка е хваната във въздуха, преди да е докоснала земята или ограда; топката е подадена в база, към която бяга противников играч и стига преди играча; бягащ играч или играч извън досег с база, е докоснат с топката или с ръкавица, в която има топка.\\ 
Pitcher &Питчер - Хвърлящият топката\\ 
Rundown &Ръндаун - положение, при което играч на отбора в нападение е хванат между две бази и е разиграван и гонен от играчи на отбора в защита, за да го докоснат - изгорят, т. е. да се отбележи аут.\\ 
Runner &Рънер - играч в нападение, който е удрял и е стигнал до някоя база и чака да бъде избутан, за да напредне от следващите го съотборници\\ 
Safe &Сейф – Положение, при което играч в нападение е достигнал в следствие на свой удар (или на удар на играч от неговия отбор, удрящ след него) до база преди топката и не е бил докосван (изгарян) от противников играч с топка. Докато играча е в контакт с базата, той е в безопасност, не може да бъде изгарян.\\ 
Stolen base &Кражба на база - действие, при което рънерът тръгва да бяга към следващата база, преди да изчака удара на батера.\\ 
Strike &Страйк – Хвърляне на топката от питчера, която е минала през страйк зоната; както и топка, на която батера е замахнал в опит да я удари.\\ 
Strikeout &Страйк аут – Аут (изгаряне), вследствие на това, че батера не е успял да удари 3 страйк топки.\\ 
Strike zone &Страйк зона – Зона, определена от разстоянието от колената до лактите на батера и над хоума.\\ 
Strike &Страйк - "Добро" хвърляне на питчера\\ 
Strike zone &Страйк зона\\ 
Time &Тайм – Прекъсване на играта поради някаква причина. Обявява се от съдията по искане на някой играч или треньор.\\ 
Top of the inning &Началото на ининга\\ 
Triple &Трипъл – Добър удар, при който батера успява безпроблемно да напредне до 3-та база.\\ 
Шорт-стоп (англ. Shortstop, съгращение SS) & игрова позиция в бейзбола. Шорт-стоп се нарича играчът, който пази отбора, намиращ се между втора и трета база.\\
\hline
\end{longtable}

\section{Бокс}

\begin{longtable}{|p{0.3\textwidth}|p{0.7\textwidth}|}
\hline
\textbf{Термин} & \textbf{Описание} \\
\hline
\endhead
\hline
\endfoot
\hline
\endlastfoot
\hline
Infighter&Инфайтър (боец от близка дистанция)\\ 
Outfighter&Аутфайтър (боец от далечна дистанция)\\ 
Infight&Вътрешен бой\\ 
Outfight&Външен бой\\ 
Jab&Джаб, рязък удар (Джаб ударът се изпълнява с водещата ръка. Лично ние сме го писали в наш превод и като ляв прав, което не смятаме, че е грешка, тъй като на героя именно лявата ръка му беше водеща.)\\ 
Flicker jab&Бърз/подготвящ джаб, шибащ удар\\ 
Solid-puncher&Стабилен удряч (Ударът на боксьора е стабилен и добре изпълнен в техническо отношение. Пример за това е левият прав на Кубрат Пулев)\\ 
Hard-puncher&Силен удряч (Боксьорът притежава силен/тежък удар. Типичен пример за това са Майк Тайсън и Джордж Форман.)\\ 
Uppercut&Ъперкът\\ 
Short uppercut&Къс ъперкът\\ 
Cross&Кръстосан удар, крос. Кросът се нарича още “десен прав” или само “десен”, особено ако не е пуснат над левия прав на противника.\\ 
Cross-counter&Контриращ прав\\ 
Crisscross&Пресичане (контриране на контраудар с контраудар)\\ 
Hook&Кроше, среща се и като хук, но аз лично не бих препоръчал такъв превод.\\ 
Guard&Гард\\ 
Side-step&Сайд степ/сайд степове\\ 
Hit and away&Атака и отклон, влизаш-удряш ( или влизаш-удряш и излизаш. Примерно, влизаш с ляв прав и после се оттегляш с отстъпване, отклон или както можеш.)\\ 
The Smash&Мачкащ удар, мачкащи удари, разбиващи удари\\ 
Counterpunch&Ответен удар, контраудар\\ 
Straight punch&Прав удар\\ 
Slip/Slipping&Отклон (Понякога се отнася и просто за подхлъзване на ринга, както е и самият превод на думата)\\ 
Right straight&Десен прав\\ 
Bob and Weave&Ескиваж\\ 
Bolo punch&Боло удар\\ 
Ducking/Duck or Break&Потапяне\\ 
Pinpoint blow&Прецизен или точен удар\\ 
Roundhouse punch&Махов удар\\ 
Slugger или brawler&Боксьор бияч, нарича се само "бияч" (така наричат и Роки Балбоа от филмите "Роки")\\ 
Southpaw&Обратен гард\\ 
K.O./Knockout&Нокаут\\ 
Knock-down&Нокдаун\\ 
Down&Даун\\ 
Break&Брейк (Разделете се!)\\ 
Shadow boxing&Бой със сянка\\ 
Sparring&Спаринг\\ 
Step work/Footwork&Работа с крака\\ 
Ring&Ринг\\ 
Gong/Bell&Гонг\\ 
Cutman&Майстор по порези, среща се и просто като "кътмен"\\ 
Cornerman&треньор, помощник, асистент (този израз се използва и за тримата помощници в ъгъла на състезателя, но най-често се отнася за треньора. Чувал съм го и като "корнърмен")\\ 
Seconds out&Секунданти, вън, ринг свободен (което означава, че на ринга остават само боксьорите и реферът.)\\ 
Mitt-punching&Лапи, бой с лапи\\ 
Sandbag&Круша, боксов чувал\\ 
Shift weight&Пренасяне на тежестта\\ 
Dash and dart&Нападателна скорост, спринт\\ 
Dempsey roll&Демпси рол\\ 
One-two&Раз-два (Комбинация от ляв и десен прав.)\\ 
Neutral corner&Неутрален ъгъл\\ 
Flyweight&Категория муха\\ 
Bantamweight&Категория петел\\ 
Junior featherweight или позната още като super bantamweight&Категория младши перо или категория супер петел\\ 
Featherweight&Категория перо, полулека\\ 
Junior lightweight или позната още като super featherweight&Категория супер перо\\ 
Lightweight&Лека категория\\ 
Lightwelterweight или junior welterweight, super lightweight&Лека полусредна категория\\ 
Welterweight&Полусредна категория\\ 
Lightmiddleweight&Лека средна категория\\ 
Middleweight&Средна категория\\ 
Lightheavyweight&Полутежка категория\\ 
Heavyweight&Тежка категория\\ 
Hitman style&Убийствен стил (терминът идва от боксьора Томас "Убиеца" Хърнс)\\ 
Cross block/guard&Кръстосан гард\\ 
Peek-a-Boo&Пийкабу (стил на защита, използван от Майк Тайсън)\\ 
Philly Shell or Crab&"Фили Шел" или Краб (стил на защита, който се използва от бойци, искащи да нанасят ответни удари)\\ 
The Cover-Up&Глуха защита\\ 
Feint&Залъгване, финт\\ 
Weigh-in&Претегляне, мерене (наричат го и просто "кантар")\\ 
Mouthpiece&Гума за уста, протектор за уста, назъбник или просто "гума"\\ 
Groggy&гроги (използва се, когато боксьорът е замаян от удари)
\end{longtable}

\section{Футбол}

\begin{longtable}{|p{0.3\textwidth}|p{0.7\textwidth}|}
    \hline
    \textbf{Термин} & \textbf{Описание} \\
    \hline
    \endhead
    \hline
    \endfoot
    \hline
    \endlastfoot
    \hline
    Attacker&Нападател\\ 
Back Heel&Петичка\\ 
Back Pass&Пас назад\\ 
Ball Carrier&Играчът с топката\\ 
Bending the Ball&Фалцов удар\\ 
Bicycle Kick&Задна ножица\\ 
Center Spot&Център\\ 
Confederation&Федерация (като УЕФА например)\\ 
Corner Flag&Флагчето за корнер / Ъглово флагче\\ 
Football Dribbler&Дрибльор\\ 
Corner Kick&Ъглов удар\\ 
Cross&Може да е подаване, центриране, пак в зависимост от ситуацията\\ 
Defender&Защитник/Бранител\\ 
Direct Free Kick&Пряк свободен удар\\ 
Dribble&Дриблиране\\ 
Dummy Run&Подлъгващо движение, финт, финтиране\\ 
Far Post&Втора греда, далечна греда\\ 
Foul&Нарушение, фал\\ 
Free Kick&Свободен удар\\ 
Futsal&Футзал\\ 
Give and Go&Двойно подаване, двоен пас, извеждаща комбинация (Разиграване на топката между двама играчи)\\ 
Goal Area&Наказателно поле\\ 
Goal Kick&Биене на аут\\ 
Goal Line&Голлиния\\ 
Goal Mouth&Вратарско поле\\ 
Goalkeeper&Вратар\\ 
Header&Подаване с глава, удар с глава\\ 
Indirect Free Kick&Непряк свободен удар\\ 
Inswinger&Фалцов удар\\ 
Kickoff&Започване от центъра, начален удар, център.\\ 
Man to Man Marking&Персонална защита\\ 
Midfielder&Халф/Полузащитник (има доста халфове-опорен/дефанзивен халф, атакуващ халф, плеймейкър, който играе зад нападателя в зависимост от схемата на игра)\\ 
Nearpost&Първа греда, близка греда.\\ 
Obstruction&Препречване\\ 
Offside Trap&Изкуствена засада\\ 
Offside&Засада\\ 
One-Touch Pass&Подаване с едно докосване\\ 
Out Swinger&Изчистване\\ 
Penalty Spot&Точка на дузпа\\ 
Penalty&Дузпа\\ 
Pitch&Терен\\ 
Red Card&Червен картон\\ 
Referee&Съдия\\ 
Shoot&Удар\\ 
Sliding Tackle&Пресичащ шпагат\\ 
Striker&Централен нападател\\ 
Sweeper&Стопер (последният защитник, който стои пред вратаря), централен защитник\\ 
Tackle&Това си просто отнемане на топката без шпагати и без футболиста, който отнема топката да пада на земята.\\ 
Through Pass&Извеждащо подаване/Извеждащ пас\\ 
Throw-In&Тъч\\ 
Toe Poke&Боц, набоцване, шутиране на топката\\ 
Touch Line&Тъч линия\\ 
Trapping the Ball&Гардиране на топката\\ 
Volley&Воле, удар от движение\\ 
Wingers&Крила (Left winger-Ляво крило/Right winger-Дясно крило)\\ 
Yellow Card&Жълт картон\\ 
Zone Defense&Зонова защита

\end{longtable}

